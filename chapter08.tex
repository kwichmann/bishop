\documentclass[12pt, a4paper]{article}
\usepackage{amsmath}
\usepackage{amsfonts}
\usepackage{amsthm}
\usepackage{mathtools}
\usepackage{tikz}
\usetikzlibrary{bayesnet}
\newtheorem{theorem}{Theorem}[section]
\newtheorem{definition}{Definition}[section]
\numberwithin{equation}{section}
\usepackage{pgfplots}
\pgfplotsset{width=10cm,compat=1.9}
\graphicspath{ {img/} }
\DeclareGraphicsExtensions{.png}

\title{Chapter 8 - Graphical models}
\author{Kristian Wichmann}

\begin{document}
\maketitle

\section{Exercise 8.1}
The parent node factorization for a Bayesian network is:
\begin{equation}
p(\mathbf{x})=\prod_i p(x_i|\textrm{pa}_i)
\label{parent_node_factorization}
\end{equation}
Show that if each of the conditional distributions are normalized, then so is the joint distribution.

\subsection{Solution}
The assumption is, that for any combination of values of $\textrm{pa}_i$, we have:
\begin{equation}
\int p(x_i|\textrm{pa}_i)\ d\mu_i(x_i)=1
\label{normalization_assumption}
\end{equation}
Here $\mu_i$ is the relevant dominating measure for the probability density function for $x_i$.

We wish to calculate:
\begin{equation}
\int p(\mathbf{x})\ d\mu_1(x_1)\cdots d\mu_K(x_K)
\end{equation}
Using equation \ref{parent_node_factorization} this can be written:
\begin{equation}
\int p(x_K|\textrm{pa}_K)\cdots p(x_1|\textrm{pa}_1) d\mu_1(x_1)\cdots d\mu_K(x_K)
\end{equation}
Now using the assumption from equation \ref{normalization_assumption} for $i=1$, the inner innermost integration turns to one. Doing so for $i=2$ and so on up to $i=K$, we get that the entire integral is equal to 1 as desired.

\section{Exercise 8.2}
Show that the property of there being no directed cycles in a directed graphs follows from the statement that there exists an ordering of the nodes, such that there are no links to a lower-numbered node.

\subsection{Solution}
Let $x_1, x_2,\dots, x_K$ be such an ordering. Assume that a directed cycle exists
\begin{equation}
x_{i_1}\rightarrow x_{i_2}\rightarrow\cdots\rightarrow x_{i_N}=x_{i_1}
\end{equation}
By the ordering assumption we must have:
\begin{equation}
i_1<i_2<\cdots<i_n
\end{equation}
But since $i_1=i_n$ we have $i_1<i_1$, which is a contradiction.

\end{document}